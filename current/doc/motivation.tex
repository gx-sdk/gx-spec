\chapter{Motivation}

These days, the Sega Genesis/Mega Drive is, on a global scale, mostly
irrelevant, and for those places where it's {\em still} relevant, there
are already tools and experts available for crafting Genesis games. This
raises the question of why somebody would bother with designing and
implementing an SDK for it.

\begin{enumerate}

{\bf \item The Sega Genesis homebrew scene is relatively small.} To
be fair, not much digging was done before reaching this conclusion,
but it's pretty clear from wandering around IRC and other such places
that of the classic consoles people just aren't as interested in the
Genesis as they are consoles like the NES or SNES. Although there are
people making Genesis games, the number of people making homebrew games
for other consoles is clearly larger. Creating an SDK would lower the
barrier to entry for Genesis homebrew development, helping create a more
vibrant homebrew scene.

{\bf \item Current toolchains are lacking.} Homebrew Genesis games
are often built with a combination of assembly and C, much like was
probably done in game studios in the Genesis' heyday. This gives the
programmer complete control over all aspects of the system, which allows
them to squeeze out every bit of performance possible. However, modern
C compilers are often designed for single-target systems, so taking
advantage of the Genesis Z80 coprocessor is a lot more work than it
really needs to be. Fighting the compiler's optimizer is another real
problem, and optimizing things like VDP writes still needs to be done
by hand. The whole process is very rigid, fragile, and manual, and not
conducive to fast-turnaround homebrew development.

{\bf \item It's a non-trivial personal challenge.} Attempting to build
a toolchain for a hardware system is not something you get done in a
couple weekends, and it's not something you can write after your first
run through {\em Learn Java in 30 days}. It's an exciting opportunity
to apply some of the many skills I've gathered over the years, and
to ultimately create a homebrew game. If \gx{} turns into something
functional, I'll consider it a success.

\end{enumerate}
