\section{Storage}

\gx{} gives the programmer a significant amount of control over how values
are laid out in memory, as this is essential to writing effective Genesis
ROMs. A number of qualifiers can be attached to a storage declaration
that determine where in memory the value will be stored, and what sorts
of properties it has.

\subsection{Random Access Memory (RAM)}

This is where variables are located by default. Values in this region
can be read from or written to. The programmer may force a value be
stored in RAM by using the {\tt ram} keyword.

Many times it is useful to be able to reuse different parts of RAM. An
example use case might be state-specific chunks of memory, such as
per-level in a game. For this, \gx{} provides ``RAM regions''. A
RAM ``region'' is a named section of RAM with 0 or more named
``layers''. Variables can be assigned to each layer as appropriate. Layers
will be overlapped in RAM as if they were members of a giant C {\tt
union}. Regions will never overlap other allocated regions of RAM.

\subsection{Read-Only Memory (ROM)}

The compiler may choose to consider variables that can proven read-only
as ROM variables. The programmer may also specify that a variable is a
ROM variable by using the {\tt rom} keyword. ROM variables are read-only;
an attempt to write to one is an error and will prevent compilation.

The layout of variables in ROM can be important to a given application. In
addition to the standard alignment and direct addressing layout directives
present in most assemblers and as common C compiler extensions, \gx{}
provides ``ROM frames''. A ROM frame specifies how many of the least
significant bits of the addresses allocated to the variable are allowed
to change. In other words, this can be used to prevent a variable from
crossing certain address boundaries.

\subsection{Memory-Mapped Peripherals}

TODO (this is the part where i re-implement {\tt volatile})
