\section{Language ABI}
\label{sec:abi}

This section specifies anything related to the compiled interface of
the language, including calling conventions and type implementations,
so that programmers can safely write C and assembly extensions to their
\gx{} code.

\subsection{Calling Convention}

All semantic function/method calls map to this calling convention. A
basic call is a sequence of 0 or more arguments, and 0 or more return
values. How the language features map down onto this structure are
described in the later parts of this subsection.

The calling convention itself is not designed to be as fast as
possible. The optimizer features an aggressive inliner which attempts
to reduce the number of actual function calls used without making the
final program text too large.

\subsubsection{68000 {\tt gx} Calls}

Arguments are passed in registers, starting from lower-numbered registers,
moving toward higher-numbered ones. Argument allocation spills to the
stack, with the leftmost spilled arguments on the top of the stack.
Whether an argument is allocated to the {\tt a0-5} or {\tt d0-7} register
range depends on whether the particular argument is a-type or d-type.
See \ref{subsec:abitypes} for more information on a-type and d-type.

Calling is performed with either the {\tt BSR} or {\tt JSR}
instruction. Returning is performed with {\tt RTS}. {\tt a6} performs the
function of frame pointer, and the {\tt LINK} and {\tt UNLK} instructions
are used, when neccessary, for establishing and releasing stack frames.

As a simple example, the following sequence of argument types has the
given argument register assignments:

\begin{center}\begin{tabular}{r|c|l}
offset  & type        & register  \\ \hline
0       & {\tt  u32}  & {\tt d0}  \\
1       & {\tt  u32}  & {\tt d1}  \\
2       & {\tt  *u8}  & {\tt a0}  \\
3       & {\tt  u16}  & {\tt d2}  \\
4       & {\tt *s32}  & {\tt a1}
\end{tabular}\end{center}

Return values are allocated in the same manner. However, since most
functions will have at most 1 return value, this means the return value
will only ever be {\tt d0} or {\tt a0}, again, depending on whether the
return value is d-type or a-type.

The callee is responsible for saving any registers not used for arguments
or return values. This way the callee doesn't need to waste time if no
extra registers are needed.

\subsection{Type Implementations}
\label{subsec:abitypes}

There are d-types and a-types. d-types are anything not to be
dereferenced. a-types are anything that would need to be dereferenced,
such as the base pointer for an array, or a pointer to a struct. There
are clear rules for deciding if a value is a-type or d-type, which I
will describe here eventually.
