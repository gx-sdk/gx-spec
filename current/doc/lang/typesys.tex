\section{Type System}

\gx{} is a strongly statically-typed language, meaning all conversions
between types must be explicit, and the type of every value is known at
compile time. The in-memory representation of values is well-specified,
and conversion to and from basic integer types is made available whenever
possible.

\subsection{Basic Type Summary}

The following basic types are built in to the language.

\begin{enumerate}

\item {\tt u8}, {\tt u16}, {\tt u32} -- Unsigned integer values
of 8, 16, and 32 bits respectively.

\item {\tt s8}, {\tt s16}, {\tt s32} -- Signed integer values of 8,
16, and 32 bits respectively.

\item {\tt bool} -- A single binary value, with the value {\tt true}
or {\tt false}. If many boolean values appear next to each other, the
compiler may choose to have them packed as individual bits as compactly as
possible in a way that does not make significant performance sacrifices.

\end{enumerate}

The following types are built in to the language and have parameters that
affect their behavior. These were designed to take advantage of certain
processor features or to wrap common user types in an expressive way.

\begin{enumerate}

\item {\tt *}{\em target} -- A pointer type. Similar to the C construct
of the same name, values of this type hold the location of a value of
type {\em target}.

\item {\tt bcd<}{\em N}{\tt >} -- Binary-coded decimal (BCD) value of
{\em N} decimal digits. Values of this type require at least $N \times
4$ bits of memory. Please see Appendix TODO for potential issues when
working with BCD types.

\item {\tt fixed<}{\em M}{\tt,}{\em N}{\tt >} -- Fixed point arithmetic
value with {\em M} bits of whole part and {\em N} bits of fractional
part. Values of this type require at least $M + N$ bits of memory. Please
see Appendix TODO for potential issues when working with BCD types.

\end{enumerate}

\subsection{Compound Type Summary}

The following compound type constructs are available.

\begin{enumerate}

\item {\tt struct} {\em name} {\tt \{} {\em members \dots} {\tt \}}
-- Similar to the C construct of the same name, {\tt struct} places its
members consecutively in memory.

\item {\tt enum} {\em name} {\tt \{} {\em variants \dots} {\tt \}} --
Similar to the Rust construct of the same name, {\tt enum} is \gx{}'s
tagged union type. It should be noted that {\tt enum}s are occasionally
more memory efficient than the equivalent C, since many instances of
an {\tt enum} will only require enough space for the active variant,
such as constants and certain instances of local variables.

\item {\tt bitvec} {\em name} {\tt (} {\em fields \dots} {\tt )} -- A
bit vector. This is essentially a bit-packed version of {\tt struct},
with much tighter control over field sizes. {\tt bitvec}s with fewer
than 32 bits are as efficient as {\tt u32}s in many cases. The goal is to
replace tedious masking and shifting needed when working with bit-packed
values. In many cases the compiler can detect effective optimization
opportunities for making {\tt bitvec} field accesses very fast. See
Appendix TODO for tips on getting the most out of {\tt bitvec}s.

\item {\tt bitpat} {\em name} {\tt (} {\em pattern} {\tt )} -- A type that
scrambles logical bits to get their physical ordering. This can be useful
for certain kinds of encoder and decoder. The goal is to replace tedious
masking and shifting needed when working with bit-packed values. See
Appendix TODO for tips on getting the most out of {\tt bitpat}s.

\end{enumerate}
