\section{Control Structures}

\gx{} provides a number of wrappers around common assembly language
control flow patterns. The key here is to encourage optimizability,
and to make certain patterns less tedious and error-prone.

\subsection{Branching}

The following branching structures allow the programmer to conditionally
redirect the flow of code.

\begin{enumerate}

\item {\tt if (}{\em condition}{\tt ) \string{} {\em body} {\tt \string}}
-- Execute {\em body} only when the expression {\em condition} has a
non-zero, non-false value.

\item {\tt if (}{\em condition}{\tt ) \string{} {\em body1} {\tt \string}
else \string{} {\em body2} {\tt \string}} -- Execute {\em body1} when {\em
condition} has a non-zero, non-false value, otherwise execute {\em body2}.

\item {\tt switch (}{\em expression}{\tt ) \{ case} {\em value}{\tt :}
{\em body ...} {\tt \}} -- Evaluates the given expression and follows
the first matching branch. See appendix TODO for tips on getting the
most out of {\tt switch} blocks.

\end{enumerate}

\subsection{Looping}

The following looping structures allow the programmer to repeat sections
of code in various vays.

\begin{enumerate}

\item {\tt loop \string{} {\em body} {\tt \string}} -- Repeats the given
code section forever.

\item {\tt while (}{\em condition}{\tt ) \string{} {\em body} {\tt
\string}} -- Execute {\em body} only while the expression {\em condition}
has a non-zero, non-false value.

\item {\tt for (}{\em identifier} {\tt in} {\em iterable ...}{\tt )
\string{} {\em body} {\tt \string}} -- Repeats the given section of code
for every item in the given {\em iterable}, with the value assigned to
{\em identifier}.  As the most flexible and common looping structure,
a wide variety of optimizations have gone into making {\tt for} loops
efficient while remaining expressive. See Appendix TODO for tips on
getting the most out of {\tt for} loops.

\end{enumerate}
