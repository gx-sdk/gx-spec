\section{Syntax}

This section describes the syntax of the \gx{} language.

\subsection{Top-Level Structure}

\begin{grammar}
\grRule{file}{\grRepeat{\grVar{unit}}}
\grRule{unit}{\grTok{unit} \grLit{\string{}
              \grRepeat{\grVar{decl}} \grLit{\string}}}
\end{grammar}

A single input file is a list of units. A \gx{} program with multiple
input files is treated as a single concatenated file.

\subsection{Declarations}

\begin{grammar}
\grRule{decl}{\grVar{decl-scope} \grVar{decl-body}}
\grRule{decl-scope}{\grTok{pub}}
\grRuleCont{\grEmpty}
\grRule{decl-body}{\grVar{type-decl}}
\grRuleCont{\grVar{storage-decl}}
\grRuleCont{\grVar{func-decl}}
\end{grammar}

\subsection{Types}

\begin{grammar}
\grRule{type-decl}{\grTok{type} \grVar{id} \grLit{:} \grVar{type-spec} \grLit{;}}
\end{grammar}

A \grVar{type-decl} is a way of creating a new type name as an alias
for another type.

\begin{grammar}
\grRule{type-spec}{\grVar{id}}
\grRuleCont{\grLit{*} \grVar{type-spec}}
\grRuleCont{\grLit{[} \grVar{number} \grLit{]} \grVar{type-spec}}
\grRuleCont{\grTok{bcd} \grLit{<} \grVar{number} \grLit{>}}
\grRuleCont{\grTok{fixed} \grLit{<} \grVar{number}
            \grLit{,} \grVar{number} \grLit{>}}
\grRuleCont{\grTok{struct} \grLit{\string{}
            \grRepeat{\grVar{storage-decl}} \grLit{\string}}}
\grRuleCont{\grVar{bitvec-spec}}
\end{grammar}

A \grVar{type-spec} specifies a type.

\begin{grammar}
\grRule{bitvec-spec}{\grTok{bitvec} \grLit{(} \grVar{bitvec-body} \grLit{)}}
\grRule{bitvec-body}{\grVar{bitvec-body$_2$}}
\grRuleCont{\grEmpty}
\grRule{bitvec-body$_2$}{\grVar{bitvec-body$_2$} \grLit{,} \grVar{bitvec-member}}
\grRuleCont{\grVar{bitvec-member}}
\grRule{bitvec-member}{\grVar{binary-string}}
\grRuleCont{\grVar{id} \grLit{:} \grVar{number}}
\end{grammar}

A \grVar{bitvec-spec} describes a bitvector type. As a
\grVar{bitvec-member}, a \grVar{binary-string} specifies a literal string
of bits to put into instances of the bitvector. An \grVar{id} specifies
a named region of bits, where \grVar{number} specifies the number of bits.

\subsection{Functions}

\begin{grammar}
\grRule{func-decl}{\grVar{func-heading}
                   \grLit{\string{} \grVar{func-body} \grLit{\string}}}
\grRule{func-heading}{\grTok{fn} \grVar{id} \grLit{(}
                      \grVar{func-params} \grLit{)} \grLit{:} \grVar{type-spec}}
\grRule{func-params}{\grVar{func-params$_2$}}
\grRuleCont{\grEmpty}
\grRule{func-params$_2$}{\grVar{func-params$_2$} \grLit{,} \grVar{func-param}}
\grRuleCont{\grVar{func-param}}
\grRule{func-param}{\grVar{id}}
\grRuleCont{\grVar{id} \grLit{:} \grVar{type-spec}}
\grRule{func-body}{\grVar{stmt-list}}
\end{grammar}

\subsection{Statements}

\begin{grammar}
\grRule{stmt-list}{\grVar{stmt-list} \grVar{stmt}}
\grRuleCont{\grEmpty}
\grRule{stmt}{\grVar{expr} \grLit{;}}
\grRuleCont{\grLit{\string{} \grVar{stmt-list} \grLit{\string}}}
\grRuleCont{\grVar{if-stmt}}
\grRuleCont{\grVar{switch-stmt}}
\grRuleCont{\grVar{loop-stmt}}
\grRuleCont{\grVar{while-stmt}}
\grRuleCont{\grVar{for-stmt}}
\grRuleCont{\grTok{break} \grLit{;}}
\grRuleCont{\grTok{continue} \grLit{;}}
\grRuleCont{\grTok{repeat} \grLit{;}}
\end{grammar}

This is the top level statement structure.

\begin{grammar}
\grRule{if-stmt}{\grTok{if} \grLit{(} \grVar{expr} \grLit{)} \grVar[1]{stmt}}
\grRuleCont{\grTok{if} \grLit{(} \grVar{expr} \grLit{)} \grVar[1]{stmt}
            \grTok{else} \grVar[2]{stmt}}
\end{grammar}

If \grVar{expr} evaluates to a non-zero value, then \grVar[1]{stmt}
is executed. Otherwise, \grVar[2]{stmt} is executed, if present.

\begin{grammar}
\grRule{switch-stmt}{\grTok{switch} \grLit{(} \grVar{expr} \grLit{)}
                     \grLit{\string{} \grVar{switch-body} \grLit{\string}}}
\grRule{switch-body}{\grVar{switch-body} \grVar{switch-case}}
\grRuleCont{\grEmpty}
\grRule{switch-case}{\grTok{case} \grVar{constant} \grLit{:} \grVar{stmt-list}}
\grRuleCont{\grTok{default} \grLit{:} \grVar{stmt-list}}
\end{grammar}

Evaluates \grVar{expr} and begins executing the statements in the switch
starting at the one with a matching \grVar{switch-case}. Note that
switch cases default to fallthrough to the next switch case, unless a
\grTok{break} is encountered during execution.

\begin{grammar}
\grRule{loop-stmt}{\grTok{loop} \grVar{stmt}}
\end{grammar}

Repeats \grVar{stmt} indefinitely.

\begin{grammar}
\grRule{while-stmt}{\grTok{while} \grLit{(} \grVar{expr} \grLit{)} \grVar{stmt}}
\end{grammar}

Repeats \grVar{stmt} only as long as \grVar{expr} evaluates to a
non-zero value.

\begin{grammar}
\grRule{for-stmt}{\grTok{for} \grLit{(} \grVar{id} \grTok{in} \grVar{expr}
                  \grLit{)} \grVar{stmt}}
\end{grammar}

Executes \grVar{stmt} once for every item in \grVar{expr}, which should
represent an iterable value. The item is assigned to a variable with
the given \grVar{id}.
