\section{Syntax}

This section describes the syntax of the \gx{} language.

\subsection{Top-Level Structure}

\begin{grammar}
\grRule{file}{\grRepeat{\grVar{unit}}}
\grRule{unit}{\grTok{unit} \grLit{\{} \grRepeat{\grVar{decl}} \grLit{\}}}
\end{grammar}

A single input file is a list of units. A \gx{} program with multiple
input files is treated as a single concatenated file.

\subsection{Declarations}

\begin{grammar}
\grRule{decl}{\grVar{decl-scope} \grVar{decl-body}}
\grRule{decl-scope}{\grTok{pub}}
\grRuleCont{\grEmpty}
\grRule{decl-body}{\grVar{type-decl}}
\grRuleCont{\grVar{storage-decl}}
\grRuleCont{\grVar{func-decl}}
\end{grammar}

\subsection{Types}

\begin{grammar}
\grRule{type-decl}{\grTok{type} \grVar{id} \grLit{:} \grVar{type-spec} \grLit{;}}
\end{grammar}

A \grVar{type-decl} is a way of creating a new type name as an alias
for another type.

\begin{grammar}
\grRule{type-spec}{\grVar{id}}
\grRuleCont{\grLit{*} \grVar{type-spec}}
\grRuleCont{\grLit{[} \grVar{number} \grLit{]} \grVar{type-spec}}
\grRuleCont{\grTok{bcd} \grLit{<} \grVar{number} \grLit{>}}
\grRuleCont{\grTok{fixed} \grLit{<} \grVar{number}
            \grLit{,} \grVar{number} \grLit{>}}
\grRuleCont{\grTok{struct} \grLit{\{}
            \grRepeat{\grVar{storage-decl}} \grLit{\}}}
\grRuleCont{\grVar{bitvec-spec}}
\end{grammar}

A \grVar{type-spec} specifies a type.

\begin{grammar}
\grRule{bitvec-spec}{\grTok{bitvec} \grLit{(} \grVar{bitvec-body} \grLit{)}}
\grRule{bitvec-body}{\grVar{bitvec-body} \grLit{,} \grVar{bitvec-member}}
\grRuleCont{\grEmpty}
\grRule{bitvec-member}{\grVar{binary-string}}
\grRuleCont{\grVar{id} \grLit{:} \grVar{number}}
\end{grammar}

A \grVar{bitvec-spec} describes a bitvector type. As a
\grVar{bitvec-member}, a \grVar{binary-string} specifies a literal string
of bits to put into instances of the bitvector. An \grVar{id} specifies
a named region of bits, where \grVar{number} specifies the number of bits.
